\documentclass[a4paper, openany]{book}
\setcounter{tocdepth}{4}
\setcounter{secnumdepth}{4}
\usepackage{graphicx}
\usepackage{amsmath,amssymb, amsfonts, geometry, float, listings, enumerate, multicol}
\usepackage{multicol, float, color, colortbl}
\usepackage{lipsum}
\usepackage{tikz, titlesec, parskip}
\usepackage{tikz,pgfplots, circuitikz}
\usepackage{graphicx}
\usepackage{subcaption}
\usepackage{spreadtab}
\usetikzlibrary{arrows, decorations.markings}
\usepackage{tikz}
\usetikzlibrary{shapes,arrows,positioning,calc}
\usepackage{xcolor, soul} 

\pgfplotsset{compat=1.5.1}
\usepgfplotslibrary{fillbetween}
\usepackage{caption, enumitem}
\usepackage{bm}
\usepackage[export]{adjustbox}
\usepackage{mathtools}
\tikzstyle{block} = [draw, fill=white, rectangle, 
    minimum height=3em, minimum width=6em]
    
\tikzstyle{vecArrow} = [thick, decoration={markings,mark=at position
   1 with {\arrow[semithick]{open triangle 90}}},
   double distance=1pt, shorten >= 5.5pt,
   preaction = {decorate},
   postaction = {draw,line width=1pt, white,shorten >= 4.5pt}]
   
\tikzstyle{innerBlue} = [semithick, blue,line width=1pt, shorten >= 4.5pt]

\tikzset{%
    block/.style={draw, fill=white, rectangle, 
            minimum height=2em, minimum width=3em},
    input/.style={inner sep=0pt},       
    output/.style={inner sep=0pt},      
    sum/.style = {draw, fill=white, circle, minimum size=2mm, node distance=1.5cm, inner sep=0pt},
    pinstyle/.style = {pin edge={to-,thin,black}}
}


\usepackage{etoc}
\usepackage{relsize}
\usepackage{systeme}
\usepackage[pagebackref=false,colorlinks,linkcolor=blue,citecolor=magenta]{hyperref}
\usepackage{wrapfig, blindtext}
\titlespacing{\section}{0pt}{10pt}{0pt}
\titlespacing{\subsection}{0pt}{10pt}{0pt}
\titlespacing{\subsubsection}{0pt}{10pt}{0pt}

\usetikzlibrary{calc,patterns,through}
\newcommand{\arcangle}{%
	\mathord{<\mspace{-9mu}\mathrel{)}\mspace{2mu}}%
}


\newcommand{\code}{\texttt} 
\newcommand*{\plogo}{\fbox{$\mathcal{SUT}$}}
\usepackage{listings}



\renewcommand{\baselinestretch}{1.2}
 \geometry{
 a4paper,
 total={170mm,257mm},
 left=20mm,
 top=20mm,
 }
 \usepackage{multicol}
\usepackage{color}
\usepackage{transparent}
\setlength{\columnseprule}{1pt}
\def\columnseprulecolor{\color{blue}}

\usepackage{fancyhdr}
\pagestyle{fancy}




\makeatletter
\renewcommand{\thesection}{%
  \ifnum\c@chapter<1 \@arabic\c@section
  \else \thechapter.\@arabic\c@section
  \fi
}
\makeatother


\usepackage{eso-pic}
               \newcommand\BackgroundIm{
               \put(0,0){
               \parbox[b][\paperheight]{\paperwidth}{%
               \vfill
               \centering
               {\transparent{0.3}
               \includegraphics[height=\paperheight,width=\paperwidth,
               keepaspectratio]{images/background.png}%
               }
               \vfill
               }}}


\begin{document}
 \AddToShipoutPicture*{\BackgroundIm}

\begin{titlepage} % Suppresses displaying the page number on the title page and the subsequent page counts as page 1
	
	\raggedleft % Right align the title page
	\rule{1pt}{\textheight} % Vertical line
	\hspace{0.05\textwidth} % Whitespace between the vertical line and title page text
	\parbox[b]{0.75\textwidth}{ % Paragraph box for holding the title page text, adjust the width to move the title page left or right on the page
		
		{\Huge\bfseries CI Course Project}\\[2\baselineskip] % Title
		{\large\textit{EEG Signal Classification}}\\[4\baselineskip] % Subtitle or further description
		{\Large\textsc{Mohammad Amin Alamalhoda}} \\ % Author name, lower case for consistent small caps
		
		\vspace{0.5\textheight} % Whitespace between the title block and the publisher
		
		{\noindent Sharif University of Technology~~\plogo}\\[\baselineskip] % Publisher and logo
		}

\end{titlepage}

\pagenumbering{roman}



{
  \hypersetup{linkcolor=black}
  \tableofcontents
    \listoffigures
  \listoftables
}



\fancyhf{}

\fancyhead[R]{\includegraphics[width=0.05\textwidth]{Shariflogo.png} }
\fancyhead[L]{MohammadAmin\\ Alamalhoda}
\cfoot{(\space \space \space \space \textbf{\thepage}  \space \space \space)}

\renewcommand{\headrulewidth}{1pt}
\renewcommand{\footrulewidth}{1pt}


\newpage  
\mainmatter
\pagenumbering{arabic}

\section{Git and Project Dependencies}

\vspace{0.3cm}
\subsection{Git}
	\vspace{0.3cm}

This Project is open source and is published on Github. You can watch it using \href{https://github.com/MohammadAminAlamalhoda/EEG-Classification}{this link}.

You can use the following bash command for cloning this project:

\begin{lstlisting}[language=bash]
  $ git clone https://github.com/MohammadAminAlamalhoda/EEG-Classification
    \end{lstlisting}
  
If you don't have \code{git} installed on your device, you can use the following bash command:
\begin{itemize}
\item Linux
\begin{lstlisting}[language=bash]
  $ sudo apt-get install git
  \end{lstlisting}
  
\item MacOS

MacOS already have git installed, check its version using bash command below:

\begin{lstlisting}[language=bash]
  $ git --version
  \end{lstlisting}
If you uninstalled it, you can install it using \code{brew}:
\begin{lstlisting}[language=bash]
  $ brew install git
  \end{lstlisting}
\item Windows

You can download source code of git and make-install it using \href{https://git-scm.com/download/win}{this link}.
\end{itemize}



\subsection{Project Dependencies}
	\vspace{0.3cm}

This project needs the following stuff in order to be compiled successfully.

\begin{itemize}
\item  Python 3.7/3.8/3.9 Kernel
\item Matlab
\item PyTorch - Python Lib

\end{itemize}

You can install the required python dependencies by running:

\begin{lstlisting}[language=bash]
  $ pip install -r requirements.txt
  \end{lstlisting}

\newpage


	
\section{Datas}
	\vspace{0.3cm}
	
I loaded the datas and plotted the amplitude and STFT of average of all the channels and trials. You can see them in Figure \ref{fig:amp_stft}.


\begin{figure}[ht]
  \centering
  \begin{subfigure}[b]{0.7\linewidth}
    \includegraphics[width=\linewidth]{images/amp.png}
    \caption{Amplitude of Average of all Trials and all Channels}
  \end{subfigure}
  \begin{subfigure}[b]{0.7\linewidth}
    \includegraphics[width=\linewidth]{images/stft.png}
    \caption{STFT of Average of all Trials and all Channels}
  \end{subfigure}
  \caption{Amplitude and STFT of Average of all Trials and all Channels}
  \label{fig:amp_stft}
\end{figure}

I plotted these figures to first know more about the datas. Some times there is a very bold difference in FFT or amplitude of the signals which can be a good feature.



	\newpage


\section{Features}
	\vspace{0.3cm}



I Extracted the following features:

\subsection{Mean Frequency}


\subsection{Median Frequency}


\subsection{Total Power of Channels}

\subsection{Powers of Delta, Theta, Alpha, and Beta Frequency Bands }

\subsection{Entropy}

\subsection{Lyapunov Exponent}

\subsection{Average of Differentiate of Trials}


\subsection{Skewness}

\subsection{Kurtosis}

\newpage

\section{Fisher Score}
	\vspace{0.3cm}
	
Using fisher score, I selected good features for further steps. By calculating  fisher score of all the features, I removed the features which their fisher score was less than $2\times Avg(all\;of\;the\;fisher\;scores)$. Then using a greedy search, I selected 16 features which has the biggest ND fisher score. Using these 16 features, I performed classification using MLPs and RBFs.















\end{document}